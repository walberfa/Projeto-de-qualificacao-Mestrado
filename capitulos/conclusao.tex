\chapter{Cronograma de Atividades}
% ---
A tabela a seguir mostra o cronograma mensal previsto de atividades a serem realizadas a partir da qualificação até a defesa da dissertação. Continuidade nos experimentos se refere a testar a abordagem deste trabalho em alguns dos códigos apresentados na Fundamentação Teórica, além de verificar os resultados para uma TLB com mais posições de entrada.

Em fevereiro foi submetido um artigo à \textit{20th  IEEE International New Circuits and Systems Conference} (NEWCAS), cujo resultado da aprovação será em Abril. Espera-se obter novos resultados para submeter novo artigo ao \textit{Symposium on Integrated Circuits and Systems Design} (SBCCI) 2022.

É interessante implementar os códigos propostos em Linguagem de Descrição de Hardware (HDL, do inglês \textit{Hardware Description Language}), e a partir daí realizar testes de área e potência. Por fim, após testar outros códigos e obter resultados mais precisos sobre as vantagens de se proteger apenas os bits necessários, finalizar o texto da dissertação. O objetivo é defender a dissertação ao final do mês de Agosto.

\begin{table}[ht] 
\centering
\caption{Cronograma mensal de atividades}
\begin{tabular}{
>{\columncolor[HTML]{EFEFEF}}l |c|c|c|c|c|c}
\hline
\textbf{Atividade}     & \multicolumn{1}{l|}{\cellcolor[HTML]{EFEFEF}\textbf{mar}} & \multicolumn{1}{l|}{\cellcolor[HTML]{EFEFEF}\textbf{abr}} & \multicolumn{1}{l|}{\cellcolor[HTML]{EFEFEF}\textbf{mai}} & \multicolumn{1}{l|}{\cellcolor[HTML]{EFEFEF}\textbf{jun}} & \multicolumn{1}{l|}{\cellcolor[HTML]{EFEFEF}\textbf{jul}} & \multicolumn{1}{l}{\cellcolor[HTML]{EFEFEF}\textbf{ago}} \\ \hline
Defesa da qualificação    & x                                                         &                                                           &                                                           &                                                           &                                                           &  \\
\hline
Continuidade nos experimentos    & x                                                         &    x                                                       &                                                           &                                                           &                                                           &  \\\hline
Resultado da submissão no NEWCAS     &                                                          & x                                                         &                                                           &                                                           &                                                           &                                                                                                                 \\ \hline
Submissão de trabalho no SBCCI 2022     &                                                          & x                                                         &                                                           &                                                           &                                                           &                                                          \\ \hline
Implementação dos códigos em HDL  &                                                           &                                                          & x                                                         &                                                           &                                                           &                                                           \\ \hline
Testar implementação em simulador de FPGA   &                                                           &                                                           & x                                                         &    x                                                       &                                                           &                                                          \\ \hline
Resultados em área e potencia   &                                                           &                                                           &                                                           & x                                                         &                                                           &                       \\ \hline
Revisão da dissertação &                                                           &                                                           &                                                           &                                                           & x                                                         &                                                          \\ \hline
Defesa da dissertação              &                                                           &                                                           &                                                           &                                                           &                                                           & x                                                        \\ \hline
\end{tabular}
\end{table}