\begin{resumo}[Abstract]
 \begin{otherlanguage*}{english}
   Programs can request the memory space needed to run to the operating system.
   They receive virtual memory space, with addresses called pages. The Memory Management Unit (MMU) is responsible for managing the pages and converting these virtual addresses into physical memory addresses. Inside the MMU is located the Translation Lookaside Buffer (TLB), a cache memory that stores only the most referenced pages. Interferences known as Multiple Cell Upsets (MCUs) can affect TLB entries’ memory cells. An MCU occurrences consequence in TLBs is the higher necessity of page fault handling, which directly interferes with the system’s performance. However, even serious faults might happen, like system freezing or data corruption. To protect TLBS, different methods have been proposed over the years, including parity techniques and spatial redundancy, but many of these solutions require more memory space or computational power, overloading the system. In this context, an error correction code (ECC) that approves the principle of locality and does not add redundancy bits, reduces those failures’ impacts. This coding involves manipulating the memory address’ bits, enhancing the Hamming distance between near addresses through parity calculation between even and odd bits. The result of these calculations is allocated in the two most significant bits (MSBs) of the address, while the other bits remain unchanged. This paper proposes to explore even more the principle of locality, calculating the parities of minor sections of bits instead of using the entire word.This method only propagates the error of the least significant bits (LSBs), where the occurrence of errors may result in serious failures. This work proposed four scenarios: calculating the parity among the 4, 8, 12, and 16 LSBs in a 32-bit address. There were running ten thousand iterations to each kind of failure (single, double adjacent, and triple adjacent) in the experiments, with pseudorandom failure injection in each scenario, inserted in different positions and addresses on each iteration. Algorithm memory traces from real applications composed the benchmark of the experiments. The implemented TLB simulator uses the Least Recently Used (LRU) substitution policy and eight memory positions. After the experiments, it was indicated that protecting the less significant bits in parity calculation showed similar results to protecting the entire word, occupying up to 20\%\ of XOR gates. Futhermore, it also greatly reduces failures by protecting only the 4-LSBs, confirming that it is possible to protect the TLB by occupying less area in the synthesis of the circuit.
   \vspace{\onelineskip}
 
   \noindent 
   \textbf{Keywords}: error correction codes. multiple cell upsets. translation lookaside buffer.
 \end{otherlanguage*}
\end{resumo}